\chapter{Running the simulation}
\label{chap:run}

\section{\texttt{integrate}: Running the simulation}
\eslabel{integrate}

\begin{essyntax}
  \variant{1} integrate \var{steps}
  \variant{2} integrate set \var{method} \opt{\var{parameter}}+
\end{essyntax}

\todo{Docs missing!}
\todo{Which integrators do exist?}

\section{\texttt{change_volume}: Changing the box volume}
\eslabel[change-volume]{change_volume}

\begin{essyntax}
  \variant{1} change_volume \var{V_new} 
  \variant{2} change_volume \var{L_new} \alt{x \asep y \asep z \asep xyz}
\end{essyntax}
Changes the volume of either a cubic simulation box to the new volume
\var{V_new} or its given x-/y-/z-/xyz-extension to the new box-length
\var{L_new}, and isotropically adjusts the particles coordinates as
well. The function returns the new volume of the deformed simulation
box.

\section{Stopping particles}
\eslabel{stopParticles}
\eslabel[stop-particles]{stop_particles}

\begin{essyntax}
  \variant{1} stopParticles
  \variant{2} stop_particles
\end{essyntax}
Halts all particles in the current simulation, setting their
velocities and forces to zero. Variant \variant{2} does not provide
feedback on the execution status.

\section{\texttt{velocities}: Setting the velocities}
\eslabel{velocities}
\begin{essyntax}
  velocities \var{v_max} 
  \opt{start \var{part_id}} 
  \opt{count \var{N_T}}
\end{essyntax}
Sets the velocities of the particles with particle ID (see The part
command) between \var{part_id} and \var{part_id}+\var{N_T}
(defaults to '0' \& '[setmd npart]-\var{part_id}') to a random vector
with length in [-vmax,vmax], and returns the absolute value of the
total velocity assigned.

\section{\texttt{invalidate_system}}
\eslabel[invalidate-system]{invalidate_system}
\begin{essyntax}
  invalidate_system
\end{essyntax}
\todo{Documentation not up to date!}

Forces a system re-init which, among others, causes the integrator to
also update the forces at its beginning (instead of re-using the
values from the previous integration step).  This is particularly
necessary to ensure continuity after setting a checkpoint:
\texttt{integrate} - \texttt{set_checkpoint} - \texttt{integrate} has
only one call to \todo{???}???, while \texttt{read_checkpoint} -
\texttt{integrate} has two at the beginning of the 2nd integrate
(because loading a new system from disk typically requires
re-initializing the system), and since ??? also uses the thermostat
which in turn draws random numbers, the two situations do not end up
at the same segment of the random number sequence, all random events
will therefore slightly differ.  To prevent this, simply include a
call to invalidate_system upon setting the checkpoint (this is being
done automatically if using tcl_checkpoint_set and
tcl_checkpoint_read beginning with v1.1 of \es{}), because in that
case both scenarios will call ??? twice at the beginning of the second
integration phase thus having their random number sequences in total
sync. The C implementation is invalidate_system.



%%% Local Variables: 
%%% mode: latex
%%% TeX-master: "ug"
%%% End: 
