\chapter{Tutorial}
\label{chap:tutorial}

\section{Quick installation}

\index{configure}\index{make}

If you have the requirements (see section \vref{sec:requirements})
installed, in many cases, to compile \es{}, it is enough to execute
the following sequence of two steps in the directory where you have
unpacked the sources:
\begin{verbatim}
> configure
> make
\end{verbatim}

In some cases, \eg{} when \es{} needs to be compiled for several
different platforms or when different versions with different sets of
features are required, it might be useful to execute the commands not
in the source directory itself, but to start \texttt{configure} from
another directory (see section \vref{sec:builddir}). Furthermore, many
features of \es{} can be selectively turned on or off in the local
configuration header of \es{} (see section \vref{sec:myconfig}) before
starting the compilation with \texttt{make}.

The shell script \texttt{configure} prepares the source code for
compilation. It will determine how to use and where to find the
different libraries and tools required by the compilation process, and
it will test what compiler flags are to be used.  The script will find
out most of these things automatically.  If something is missing, it
will complain and give hints how to solve the problem.  The
configuration process can be controlled with the help of a number of
options that are explained in section \vref{sec:configure}.

The command \texttt{make} will compile the source code. Depending on
the options passed to the program, \texttt{make} can also be used for
a number of other things:
\begin{itemize}
\item It can install and uninstall the program to some other
  directories. However, normally it is not necessary to actually
  \textit{install} \es{} to run it.
\item It can test the \es{} program for correctness.
\item It can build the documentation.
\end{itemize}
The details of the usage of \texttt{make} are described in section
\vref{sec:make}.

When these steps have successfully completed, \es{} can be started
with the command (see section \vref{sec:run})
\begin{verbatim}
> Espresso
\end{verbatim}

\section{Running \es{}}
\begin{itemize}
\item interactive use (sample session)
\item small sample session
\item script execution
\end{itemize}


\section{Doing a simulation}

\begin{itemize}
\item \verb!erice_tutorial! or appendix A from paper
\item Reference to \verb!tutorial.tcl! (?)
\end{itemize}


%%% Local Variables: 
%%% mode: latex
%%% TeX-master: "ug"
%%% End: 
