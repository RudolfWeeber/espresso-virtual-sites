\chapter{Features}
\label{sec:features}
\index{features|textbf}

\newcommand{\feature}[1]{\texttt{\textbf{#1}}}

This chapter describes the features that can be activated in \es. Even
if possible, it is not recommended to activate all features, because
this will negatively effect \es's performance.

Features can be activated in the configuration header \texttt{myconfig.h} (see section
\vref{sec:myconfig}). Too activate \texttt{FEATURE}, add the following
line to the header file:
\begin{verbatim}
#define FEATURE
\end{verbatim}

\section{General features}

\begin{itemize}
\item \feature{PARTIAL\_PERIODIC} By default, all coordinates in \es{} are periodic. With
  \texttt{PARTIAL\_PERIODIC} turned on, the \es{} global variable \texttt{periodic} (see section
  \vref{sec:globalvar}) controls the periodicity of the individual coordinates. Note that this slows
  the integrator down by around $10-30\%$.
\item \feature{ELECTROSTATICS} This switches on the various electrostatics algorithms, such as
  the Ewald summation. See section \vref{sec:electrostatics} for details on this algorithms.
\item \feature{ROTATION} Switch on rotational degrees of freedom for the particles, as well as
  the corresponding quaternion integrator. See section \vref{sec:rotation} for details.
\item \feature{DIPOLES} This activates the dipole support in P$^3$M. Currently, a mixing of
  dipoles and charges is not possible, i.~e. all particles have to have charge $q=0$.
  Requires \texttt{ELECTROSTATICS} and \texttt{ROTATION}.
\item \feature{EXTERNAL\_FORCES} Allows to define an arbitrary constant force for each particle
  individually. Also allows to fix individual coordinates of particles, e.~g. keep them at a fixed
  position or within a plane.
\item \feature{CONSTRAINTS} Turns on various spatial constraints such as spherical compartments
  or walls. This constraints interact with the particles through regular short ranged potentials
  such as the Lennard--Jones potential. See section \vref{sec:constraints} for possible constraint
  forms.
\item \feature{MASS} Allows particles to have individual masses. Note that some analyzation
  procedures have not yet been adapted to take the masses into account correctly.
\item \feature{EXCLUSIONS} Allows to exclude specific short ranged interactions within
  molecules, which is necessary for some atomistic models.
\item \feature{COMFORCE}
\item \feature{COMFIXED}
\item \feature{MOLFORCES}
\item \feature{BOND\_CONSTRAINT} Turns on the RATTLE integrator which allows for fixed lengths
  bonds between particles.
\end{itemize}

In addition, there are switches that enable additional features in the integrator:
\begin{itemize}
\item \feature{NEMD} Enables the non-equilbrium (shear) MD support (see section \vref{sec:NEMD}).
\item \feature{NPT} Enables an on--the--fly NPT integration scheme (see section \vref{sec:NPT}).
\item \feature{DPD} Enables the dissipative particle dynamics thermostat (see
  section \vref{sec:DPD}).
\item \feature{LB} Enables the lattice-Boltzmann fluid code (see section \vref{sec:LB}).
\end{itemize}

\section{Interactions}

The following switches turn on various short ranged interactions (see section
\vref{sec:shortrange}):
\begin{itemize}
\item \feature{TABULATED} Enable support for user--defined interactions, e.~g. for atomistic
  simulations.
\item \feature{LENNARD\_JONES} Enable the Lennard--Jones potential.
\item \feature{LJ\_WARN\_WHEN\_CLOSE} This adds an additional check to the Lennard--Jones
  potential that prints a warning of particles come too close so that the simulation becomes
  unphysical.
\item \feature{MORSE} Enable the Morse potential.
\item \feature{LJCOS} Enable the Lennard--Jones potential with a cosine--tail.
\item \feature{LJCOS2}
\item \feature{BUCKINGHAM} Enable the Buckingham potential.
\item \feature{SOFT\_SPHERE} Enable the soft sphere potential.
\end{itemize}

If you want to use angle bonds, you currently need to choose the type
a priori (see section \vref{sec:angle}). This will change in the near
future to three independent angle potentials:
\begin{itemize}
\item \feature{BOND\_ANGLE\_HARMONIC}
\item \feature{BOND\_ANGLE\_COSINE}
\item \feature{BOND\_ANGLE\_COSSQUARE}
\end{itemize}

\section{Debug messages}

Finally, there are a number of flags for debugging. The most important one are
\begin{itemize}
\item \feature{ADDITIONAL\_CHECKS} Enables numerous additional checks which can detect
  inconsistencies especially in the cell systems. This checks are however too slow to be enabled in
  production runs.
\item \feature{MEM\_DEBUG} Enables an internal memory allocation checking system. This produces
  output for each allocation and freeing of a memory chunk, and therefore allows to track down
  memory leaks. This works by internally replacing \texttt{malloc}, \texttt{realloc} and
  \texttt{free}.
\end{itemize}

The following flags control the debug output of various sections of Espresso. You will however
understand the output very often only by looking directly at the code.
\begin{itemize}
\item \feature{COMM\_DEBUG} Output from the asynchronous communication code.
\item \feature{EVENT\_DEBUG} Notifications for event calls, i.~e. the \texttt{on\_?} functions
  in \texttt{initialize.c}. Useful if some module does not correctly respond to changes of e.~g.
  global variables.
\item \feature{INTEG\_DEBUG} Integrator output.
\item \feature{CELL\_DEBUG} Cellsystem output.
\item \feature{GHOST\_DEBUG} Cellsystem output specific to the handling of ghost cells and the
  ghost cell communication.
\item \feature{GHOST\_FORCE\_DEBUG}
\item \feature{VERLET\_DEBUG} Debugging of the Verlet list code of the domain decomposition cell
  system.
\item \feature{LATTICE\_DEBUG} Universal lattice structure debugging.
\item \feature{HALO\_DEBUG}
\item \feature{GRID\_DEBUG}
\item \feature{PARTICLE\_DEBUG} Output from the particle handling code.
\item \feature{P3M\_DEBUG}
\item \feature{ESR\_DEBUG} debugging of P$^3$Ms real space part.
\item \feature{ESK\_DEBUG} debugging of P$^3$Ms $k$--space part.
\item \feature{EWALD\_DEBUG}
\item \feature{FFT\_DEBUG} Output from the unified FFT code.
\item \feature{MAGGS\_DEBUG}
\item \feature{RANDOM\_DEBUG}
\item \feature{FORCE\_DEBUG} Output from the force calculation loops.
\item \feature{THERMO\_DEBUG} Output from the thermostats.
\item \feature{LJ\_DEBUG} Output from the Lennard--Jones code.
\item \feature{MORSE\_DEBUG} Output from the Morse code.
\item \feature{FENE\_DEBUG}
\item \feature{ONEPART\_DEBUG} Define to a number of a particle to obtain output on the forces
  calculated for this particle.
\item \feature{STAT\_DEBUG}
\item \feature{POLY\_DEBUG}
\item \feature{MOLFORCES\_DEBUG}
\item \feature{LB\_DEBUG} Output from the lattice--Boltzmann code.
\item \feature{ASYNC\_BARRIER} Introduce a barrier after each asynchronous command
  completion. Helps in detection of mismatching communication.
\item \feature{FORCE\_CORE} Causes \es{} to try to provoke a core dump when exiting
  unexpectedly.
\item \feature{MPI\_CORE} Causes \es{} to try this even with MPI errors.
\end{itemize}

%%% Local Variables: 
%%% mode: latex
%%% TeX-master: "ug"
%%% End: 
