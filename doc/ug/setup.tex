\chapter{Setting up the system}
\label{chap:setup}

\section{\texttt{setmd}: Setting global variables.}
\newescommand{setmd}

\begin{essyntax}
\variant{1} setmd \var{variable}
\variant{2} setmd \var{variable} \opt{\var{value}}+
\end{essyntax}
\todo{Explain '+' in intro.}

Variant \variant{1} returns the value of the \es global variable
\var{variable}, variant \variant{2} can be used to set the variable
\var{variable} to \var{value}. The following global variables can be
set:

%% List-environment for the description of the global variables
\newenvironment{globvar}{
  \begin{list}{}{
      \setlength{\rightmargin}{1em}
      \setlength{\leftmargin}{2em}
      \setlength{\partopsep}{0pt}
      \setlength{\topsep}{1ex}
      \setlength{\parsep}{0.5ex}
      \setlength{\listparindent}{-1em}
      \setlength{\labelwidth}{0.5em}
      \setlength{\labelsep}{0.5em}
      \renewcommand{\makelabel}[1]{%
        \index{##1@\texttt{##1} (global variable)|mainindex}%
        \index{global variables!\texttt{##1}|mainindex}%
        \texttt{##1}%
      }
    }
  }{
  \end{list}
}
\newcommand{\ro}{\emph{read-only}}

\todo{Better throw some out (\eg switches)?}
\todo{Missing: lattice_switch, dpd_tgamma, n_rigidbonds}
\todo{Which commands can be used to set the \emph{read-only}
  variables?}
\begin{globvar}
\item[box_l] (double[3]) Simulation box length.
  \todo{document what happens to the particles when \keyword{box_l} is
    changed!}
\item[cell_grid] (int[3], \ro) Dimension of the inner
  cell grid.
\item[cell_size] (double[3], \ro) Box-length of a cell.
\item[dpd_gamma] (double, \ro) Friction constant for the
  DPD thermostat.
\item[dpd_r_cut] (double, \ro) Cutoff for DPD thermostat.
\item[gamma] (double, \ro) Friction constant for the
  Langevin thermostat.
\item[integ_switch] (int, \ro) Internal switch which integrator to
  use.
\item[local_box_l] (int[3], \ro) Local simulation box length of the
  nodes.
\item[max_cut] (double, \ro) Maximal cutoff of real space
  interactions.
\item[max_num_cells] (int) Maximal number of cells for the link cell
  algorithm.  Reasonable values are between 125 and 1000, or for some
  problems (\var{n_total_particles} / \var{n_nodes}).
\item[max_part] (int, \ro) Maximal identity of a particle.
  \emph{This is in general not related to the number of particles!}
\item[max_range] (double, \ro) Maximal range of real space
  interactions: \var{max_cut} + \var{skin}.
\item[max_skin] (double, \ro) Maximal skin to be used for the link
  cell/verlet algorithm. This is the minimum of \var{cell_size} -
  \var{max_range}. \todo{???}
\item[min_num_cells] (int) \todo{???} Minimal number of cells for the
  link cell algorithm. Reasonable values range in $1e-6 N^2$ to $1e-7
  N^2$. In general just make sure that the Verlet lists are not
  incredibly large. By default the minimum is 0, but for the automatic
  P3M tuning it may be wise to larger values for high particle
  numbers.
\item[n_layers] (int, \ro) Number of layers in cell structure LAYERED
  (see section \vref{sec:cell-systems}).
\item[n_nodes] (int, \ro) Number of nodes.
\item[n_part] (int, \ro) Total number of particles.
\item[n_part_types] (int, \ro) Number of particle types that were
  used so far in the \keyword{inter} command (see chapter{tcl:inter}).
\item[node_grid] (int[3]) 3D node grid for real space domain
  decomposition (optional, if unset an optimal set is chosen
  automatically).
\item[nptiso_gamma0] (double, \ro)\todo{Docs missing.}
\item[nptiso_gammav] (double, \ro)\todo{Docs missing.}
\item[npt_p_ext] (double, \ro) Pressure for NPT simulations.
\item[npt_p_inst] (double) Pressure calculated during an
  NPT_isotropic integration.
\item[piston] (double, \ro) Mass off the box when using NPT_isotropic
  integrator.
\item[periodicity] (bool[3]) Specifies periodicity for the three
  directions. If the feature PARTIAL_PERIODIC is set, this variable
  can be set to (1,1,1) or (0,0,0) at the moment.  If not it is
  readonly and gives the default setting (1,1,1).\todo{Correct?}
\item[skin] (double) Skin for the Verlet list.
\item [temperature] (double, \ro) Temperature of the
  simulation.
\item[thermo_switch] (double, \ro) Internal variable which thermostat
  to use. 
\item[time] (double) The simulation time.
\item[time_step] (double) Time step for MD integration.
\item[timings] (int) Number of timing samples to take into account if
  set.\todo{???}
\item[transfer_rate] (int, \ro) Transfer rate for VMD connection. You
  can use this to transfer any integer value to the simulation from
  VMD.
\item[verlet_flag] (bool) Indicates whether the Verlet list will be
  rebuild. The program decides this normally automatically based on
  your actions on the data.
\item[verlet_reuse] (double) Average number of integration steps the
  verlet list has been re-used.
\end{globvar}

\section{\texttt{thermostat}: Setting up the thermostat}
\newescommand{thermostat}

The \keyword{thermostat} command is used to change settings of the
thermostat.  

The different available thermostats will be described in the following
subsections. Note that for a simulation of the NPT ensemble, you need
to use a standard thermostat for the particle velocities (\eg Langevin
or DPD), and a thermostat for the box geometry (\eg the isotropic NPT
thermostat).

You may combine different thermostats at your own risk by turning them
on one by one. Note that there is only one temperature for all
thermostats.

\subsection{Langevin thermostat}
\begin{essyntax}
  thermostat langevin \var{temperature} \var{gamma}
\end{essyntax}

The Langevin thermostat consists of a friction and noise term coupled
via the fluctuation-dissipation theorem. The friction term is a
function of the particle velocities.  \todo{Reference}

If the feature \feature{ROTATION} is compiled in, the rotational
degrees of freedom are also coupled to the thermostat.

\subsection{Dissipative Particle Dynamics (DPD) thermostat}
\begin{essyntax}
  thermostat dpd \var{temperature} \var{gamma} \var{r\_cut}
\end{essyntax}

Consists of a friction and noise term coupled via the fluctuation-
dissipation theorem. The friction term is a function of the relative
velocity of particle pairs.  The DPD thermostat is better for dynamics
than the Langevin thermostat, since it mimics hydrodynamics in the
system.  

When using a Lennard-Jones interaction, $\var{r\_cut} = 2^{\frac{1}{6}}
\sigma$ is a good value to choose, so that the thermostat acts on the
relative velocities between nearest neighbor particles.  Larger
cutoffs including next nearest neighbors or even more are unphysical.

\var{gamma} is basically an inverse timescale on which the system
thermally equilibrates.  Values between $0.1$ and $1$ are o.k, but you
propably want to try yourself to get a feeling for how fast
temperature jumps during a simulation are. The dpd thermostat does not
act on the system center of mass motion.  Therefore, before using dpd,
you have to stop the center of mass motion of your system, which you
can achieve by using the command \keyword{galileiTransformParticles}
(see section \vref{tcl:galileiTransformParticles}) . This may be
repeated once in a while for long runs due to round off errors (check
this with the command \keyword{system_com_vel} from section
\vref{tcl:system-com-vel}).

\subsection{Isotropic NPT thermostat}
\begin{essyntax}
  thermostat npt_isotropic \var{temperature} \var{gamma0} \var{gammaV}
\end{essyntax}

This theormstat is based on the Anderson thermostat and will
thermalize the box geometry. It will only do isotropic changes of the
box.\todo{Docs, reference}

\subsection{Turning off all thermostats}
\begin{essyntax}
  thermostat off
\end{essyntax}

Turns off all thermostats and sets all thermostat variables to zero.

\subsection{Getting the parameters}
\begin{essyntax}
  thermostat
\end{essyntax}

Returns the thermostat parameters. \todo{Document return format.}

\section{\texttt{nemd}: Setting up non-equilibrium MD}
\newescommand{nemd}

\index{NEMD}

\begin{essyntax}
  \variant{1}nemd exchange \var{n\_slabs} \var{n\_exchange}
  \variant{2}nemd shearrate \var{n\_slabs} \var{shearrate}
  \variant{3}nemd off
  \variant{4}nemd
  \variant{5}nemd profile
  \variant{6}nemd viscosity
\end{essyntax}
\todo{Put \texttt{nemd profile|viscosity} into \texttt{analyze}?}  
Use NEMD (Non Equilibrium Molecular Dynamics) to simulate a system
under shear with help of an unphysical momentum change in two slabs in
the system.

Variants \variant{1} and \variant{2} will initialise NEMD. Two
distinct methods exist.  Both methods divide the simulation box into
\var{n\_slab} slabs that lie parallel to the x-y-plane and apply a
shear in x direction.  The shear is applied in the top and the middle
slabs. Note, that the methods should be used with a DPD thermostat or
in an NVE ensemble.  Furthermore, you should not use other special
features like \keyword{part fix} or \keyword{constraints} inside the
top and middle slabs.

\todo{Why should NEMD only be used with DPD or in NVE?}
\todo{References for the methods}

\index{momentum exchange method}
Variant \variant{1} chooses the momentum exchange method.  In this
method, in each step the \var{n\_exchange} largest positive
x-components of the velocity in the middle slab are selected and
exchanged with the \var{n\_exchange} largest negative x-components of
the velocity in the top slab. 

\index{shear-rate method}
Variant \variant{2} chooses the shear-rate method. In this method, the
targetted x-component of the mean velocity in the top and middle slabs
are given by 
\begin{equation}
  \var{target\_velocity} = \pm \var{shearrate}\,\frac{L_z}{4}
\end{equation}
where $L_z$ is the simulation box size in z-direction. During the
integration, the x-component of the mean velocities of the top and
middle slabs are measured.  Then, the difference between the mean
x-velocities and the target x-velocities are added to the x-component
of the velocities of the particles in the respective slabs. 

Variant \variant{3} will turn off NEMD, variant \variant{4} will
return the parameters of NEMD. Variant \variant{5} will return the
velocity profile of the system in x-direction.

\todo{Describe return format of variants 4 and 5.}

Variant \variant{6} will return the viscosity of the system, that is
computed via
\begin{equation}
  \eta = \frac{F}{\dot{\gamma} L_x L_y}
\end{equation}
where $F$ is the mean force (momentum transfer per unit time) acting
on the slab and $L_x L_y$ is the area of the slab.
\todo{What is $\dot{\gamma}$?}


\section{\texttt{cellsystem}: Setting up the cell system}
\label{sec:cell-systems}
\newescommand{cellsystem}

This section deals with the flexible particle data organization of
\es.  Due to different needs of different algorithms, \es is able to
change the organization of the particles in the computer memory,
according to the needs of the used algorithms. For details on the
internal organization, refer to section
\vref{sec:internal-particle-organization}.

\subsection{Domain decomposition}
\index{domain decomposition}
\begin{essyntax}
  cellsystem domain_decomposition \opt{-no_verlet_list}
\end{essyntax}
This selects the domain decomposition cell scheme, using Verlet lists
for the calculation of the interactions. If you specify
\keyword{-no_verlet_list}, only the domain decomposition is used, but
not the Verlet lists.

The domain decomposition cellsystem is the default system and suits
most applications with short ranged interactions. The particles are
divided up spatially into small compartments, the cells, such that the
cell size is larger than the maximal interaction range. In this case
interactions only occur between particles in adjacent cells. Since the
interaction range should be much smaller than the total system size,
leaving out all interactions between non-adjacent cells can mean a
tremendous speed-up. Moreover, since for constant interaction range,
the number of particles in a cell depends only on the density. The
number of interactions is therefore of the order N instead of order
$N^2$ if one has to calculate all pair interactions.

\subsection{N-squared}
\begin{essyntax}
  cellsystem nsquare 
\end{essyntax}
This selects the very primitive nsquared cellsystem, which calculates
the interactions for all particle pairs. Therefore it loops over all
particles, giving an unfavorable computation time scaling of $N^2$.
However, algorithms like MMM1D or the plain Coulomb interaction in the
cell model require the calculation of all pair interactions.

In a multiple processor environment, the nsquared cellsystem uses a
simple particle balancing scheme to have a nearly equal number of
particles per CPU, \ie $n$ nodes have $m$ particles, and $p-n$ nodes
have $m+1$ particles, such that $n*m+(p-n)*(m+1)=N$, the total number
of particles. Therefore the computational load should be balanced
fairly equal among the nodes, with one exception: This code always
uses one CPU for the interaction between two different nodes. For an
odd number of nodes, this is fine, because the total number of
interactions to calculate is a multiple of the number of nodes, but
for an even number of nodes, for each of the $p-1$ communication
rounds, one processor is idle.

E.g. for 2 processors, there are 3 interactions: 0-0, 1-1, 0-1.
Naturally, 0-0 and 1-1 are treated by processor 0 and 1, respectively.
But the 0-1 interaction is treated by node 1 alone, so the workload
for this node is twice as high. For 3 processors, the interactions are
0-0, 1-1, 2-2, 0-1, 1-2, 0-2. Of these interactions, node 0 treats 0-0
and 0-2, node 1 treats 1-1 and 0-1, and node 2 treats 2-2 and 1-2.

Therefore it is highly recommended that you use nsquared only with an
odd number of nodes, if with multiple processors at all. 

\subsection{Layered cell system}
\begin{essyntax}
  cellsystem layered \var{n\_layers}
\end{essyntax}

This selects the layered cell system, which is specifically designed
for the needs of the MMM2D algorithm. Basically it consists of a
nsquared algorithm in x and y, but a domain decomposition along z, i.
e. the system is cut into equally sized layers along the z axis. The
current implementation allows for the cpus to align only along the z
axis, therefore the processor grid has to have the form 1x1xN.
However, each processor may be responsible for several layers, which
is determined by \var{n\_layers}, i. e. the system is split into
N*\var{n\_layers} layers along the z axis. Since in x and y direction
there are no processor boundaries, the implementation is basically
just a stripped down version of the domain decomposition cellsystem.

%%% Local Variables: 
%%% mode: latex
%%% TeX-master: "ug"
%%% End: 
