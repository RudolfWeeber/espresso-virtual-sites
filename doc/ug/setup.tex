\chapter{Setting up the system}
\label{chap:setup}

\section{\texttt{inter}: Setting up interactions}
\label{sec:inter}

\subsection{Nonbonded interactions}
\label{sec:inter_nb}
%\quickrefheading{Nonbonded interactions}

\tclcommand[LENNARD\_JONES]{inter}{%
  \var{type1 type2} 
  lennard\_jones 
  \var{epsilon sigma cutoff shift offset}
}

Defines a Lennard-Jones interaction between particles of the types
\var{type1} and \var{type2}.
\bigskip

\subsection{Bonded interactions}
\label{sec:inter_bonded}

\index{Bonded interactions} \index{Bonded interaction type id} Bonded
interactions possess an \emph{bonded interaction type id}. On the one
hand, this id is used when particles and bonds between particles are
specified in the command \texttt{part} (see section \vref{sec:part}).
On the other hand, the id is used when the interaction is specified.

\subsection{Coulomb interaction}
\label{sec:inter_electrostatics}

\subsection{Other interaction types}
\label{sec:inter_other}

\subsection{Getting the currently defined interactions}

%\quickrefheading{Getting interactions}
\tclcommand{inter}{ }

%%% Local Variables: 
%%% mode: latex
%%% TeX-master: "ug"
%%% End: 
