\chapter{Installation}
\label{chap:install}
\index{Installation|textbf}

\begin{itemize}
\item Compiling \es{} is a necessary evil
\item Features can be compiled in or not
\item For maximal efficiency, compile in only the features that you
  use
\item \es{} can be obtained from the \es{} home page
  \footnote{\url{http://www.espresso.mpg.de}}.
\end{itemize}


\section{Source and build directory}
\label{sec:builddir}
\index{build directory} \index{source directory}

If you plan to use \es{} with a single configuration, you can skip the
rest of this section. If then you have problems finding the \es{}
binary or you come upon a reference to the \emph{build directory} in
the documentation, you might have to read it, anyway. 

Usually, when a program is compiled, the resulting binary files are
put into the same directory as the sources of the program. In \es{},
the \emph{source directory} that contains all the source files can be
completely separated from the \emph{build directory} where the files
created by the build process are put. As the source directory is not
touched during the compilation process, it is possible to compile more
than one binary version of \es{} from the same set of source files.
This is useful in cases when \es{} is to be used on different computer
hardware or with a different configuration.

The source directory is the directory that contains the source files.
The location of the build directory is determined when the
\texttt{configure}-script is called.  Usually, the build directory is
assigned to the current working directory when the
\texttt{configure}-script was called. All further commands concerning
compiling and running \es{} have to be called from this directory.

\paragraph{Example}
When the source directory is \texttt{\$srcdir} (\ie{} the files where
unpacked to this directory), then the build directory can be set to
\texttt{\$builddir} by calling the \texttt{configure}-script from
there:
\begin{verbatim}
> cd $builddir
> $srcdir/configure
> make
> Espresso
\end{verbatim}

When \texttt{configure} is called directly from the source directory,
the \es{} build system is prepared to handle different platforms.  A
new subdirectory is created and \texttt{configure} is recursively
called from this directory, making the subdirectory the build
directory.  The directory is called
\texttt{obj-}\textit{platform}\texttt{/}, where \textit{platform} is a
descriptor of the CPU type where the script was started, \eg{}
\texttt{obj-Athlon\_64-pc-linux}.

In this case it is also possible to run the commands \texttt{make} and
\texttt{Espresso} directly in the source directory.

Furthermore, the option \texttt{--enable-chooser} will be set in the
recursive call of \texttt{configure} that activates the automatic
binary chooser (see section \vref{sec:install_dir}).

\section{Installation directories}
\label{sec:install_dir}

Normally, the \es-binary \texttt{Espresso-bin} is installed in the
directory \texttt{\$prefix/libexec/} and a the wrapper script
\texttt{Espresso} in the directory \texttt{\$prefix/bin/} that handles
the MPI invocation.

When the \texttt{configure}-script is called from the source directory
or when the option \texttt{--enable-chooser} is given, an automatic
binary chooser is installed in the directory \texttt{\$prefix/bin/}
and the \es{}-binary and the MPI wrapper script are installed in an
architecture-specific subdirectory
\mbox{\texttt{\$exec-prefix/lib/espresso/obj-}\textit{platform}\texttt{/}}.
When called, the binary chooser will automatically call the MPI
wrapper script in the right subdirectory.

\section{The configuration header \texttt{myconfig.h}}
\label{sec:myconfig}

\index{myconfig.h} \index{configuration header} \es{} has a great
number of features that can be compiled into the binary. However, it
is not recommended to actually compile in all possible features, as
this will negatively affect \es's performance. Instead, compile in
only the features that are actually required. For the developers, it
is also possible to turn on or off a number of debugging messages. The
features and debug messages can be controlled via a configuration
header file that contains C-preprocessor declarations. Chapter
\vref{chap:features} describes the available features.

By default, the configuration header is called \texttt{myconfig.h}.
The name of the configuration header can be either changed when the
\texttt{configure}-script is called with the option
\texttt{--with-myconfig} (see section \vref{sec:configure}), or when
\texttt{make} is called with the setting
\texttt{myconfig=}\textit{myconfig\_header} (see section
\vref{sec:make}).

The configuration header can be put in the build directory, or in the
source directory. When a configuration header is found in both
directories, the one in the build directory will be used. If both
directories do not contain a configuration header, a default header
will be used that turns on the default features.

The file \texttt{myconfig-sample.h} in the source directory contains
an example configuration header.

\paragraph{Example}
The configuration header can be used to compile different versions
from the same source directory. Suppose that you have a source
directory \texttt{\$srcdir} and two build directories
\texttt{\$builddir1} and \texttt{\$builddir2} that contain different
configuration headers:

\begin{itemize}
\item \texttt{\$builddir1/myconfig.h}:
\begin{verbatim}
#define ELECTROSTATICS
#define LENNARD-JONES
\end{verbatim}

\item \texttt{\$builddir2/myconfig.h}:
\begin{verbatim}
#define LJCOS
\end{verbatim}
\end{itemize}

\noindent Then you can simply compile two different versions of \es{} via
\begin{verbatim}
cd $builddir1
$srcdir/configure
make

cd $builddir2
$srcdir/configure
make
\end{verbatim}


\section{Running configure}
\label{sec:configure}

\index{configure}
The shell script \texttt{configure} collects all the information
required by the compilation process. It will determine how to use and
where to find the different libraries and tools required by the
compilation process, and it will test what compiler flags are to be
used.  The script will find out most of these things automatically.
If something is missing, it will complain and give hints how to solve
the problem.

The generic syntax of calling the \texttt{configure} script is:
\begin{syntax}
 $>$ configure [\var{options} ...] [\var{variable}=\var{value} ...]
\end{syntax}

\index{configure options}
The behaviour of \texttt{configure} can be controlled by the means of
command line options. In the following, only those command line
options that are special to \es{} will be explained.  For a complete
list of options and explanations thereof, call
\begin{verbatim}
> configure --help
\end{verbatim}

\begin{description}
\item [\texttt{--enable-chooser}] This option will enable the
  automatic choosing mechanism for \es{} (see section
  \vref{sec:install_dir}).  This option will be automatically enabled,
  when the \texttt{configure} script is called from the source
  directory, otherwise it will be disabled. It is not recommended to
  set the option manually.
\item[\texttt{--enable-config=KNOWN\_CONFIG}] For some known systems,
  where \texttt{configure} does not find the required libraries and
  compiler options, predefined settings can be used. The following
  configuration names are known: \texttt{dino} and \texttt{blade}. The
  default for this option is: \texttt{none}.
\item[\texttt{--enable-debug}] This option will enable compiler flags
  required for debugging \es{} and is disabled by default.
\item[\texttt{--enable-profiling}] This option will enable compiler
  flags required for profiling \es{} and is disabled by default.
\item[\texttt{--disable-processor-optimization}] This option will
  control whether \texttt{configure} will check for several
  optimization flags to be used by the compiler. This option is
  enabled by default.
\item[\texttt{--enable-xlc-qipa}] This option is only useful when the
  IBM C-compiler \texttt{xlc} is used and will control whether or not
  the compiler flag \texttt{-qipa} is used. This option is enabled by
  default.

\item[\texttt{--with-myconfig=MYCONFIG\_HEADER}] This option sets the
  name of the local configuration header (see \vref{sec:myconfig}). It
  defaults to ``\texttt{myconfig.h}''.
\item[\texttt{--with-mpi=MPI}] Sets the MPI implementation that should
  be used. By default, \texttt{configure} will test autoamtically what
  MPI implementation is available. The following implementations are
  known: 
  \begin{description}
  \item[\texttt{fake}, \texttt{no}] This will disable MPI completely.
  \item[\texttt{lam}] Use the LAM/MPI environment
    (\url{http://www.lam-mpi.org/}).
  \item[\texttt{mpich}] Use the MPICH environment
    (\url{http://www-unix.mcs.anl.gov/mpi/mpich/}).
  \item[\texttt{poe}] Use the POE environment (IBM).
  \item[\texttt{dmpi}] Use the DMPI environment (Tru64).
  \item[\texttt{generic}] Use a generic MPI implementation. This will
    try to find an MPI compiler and an MPI runtime environment.
  \end{description}
\item[\texttt{--with-efence}] Whether or not to use the ``electric
  fence'' memory debugging library
  (\url{http://freshmeat.net/projects/efence/}). Efence is not used by
  default.
\item[\texttt{--with-tcl=TCL}] When the wrong version of the Tcl
  library is used by configure, the name of the Tcl-library can be
  specified with this option, \eg{} \texttt{tcl8.4}.
\item[\texttt{--with-tk=TK}] By default, the GUI toolkit Tk is not
  used by \es. This option can be used to activate Tk and to specify
  which Tk version to use, \eg{} \texttt{tk8.4}.
\item[\texttt{--with-fftw=VERSION}] This can be used to specify which
  version of fftw is to be used. By default, version 3 will be used if
  it is found, otherwise version 2 is used.
\end{description}

\section{Compiling, testing and installing \es}
\label{sec:make}

The command \texttt{make} is mainly used to compile the \es{} source
code, but it can do a number of other things. The generic syntax of
the \texttt{make} command is:
\begin{syntax}
 $>$ make [\var{target}...] [\var{variable}=\var{value}]
\end{syntax}
When no target is given, the target \texttt{all} is used. The
following targets are available:
\begin{description}
\item[\texttt{all}] Compiles the complete \es{} source code.
\item[\texttt{check}] Runs the testsuite. By default, all available
  tests will be run on 1, 2, 3, 4, 6, or 8 processors. Which tests are
  run can be controlled by means of the variable \texttt{tests}, which
  processor numbers are to be used can be controlled via the variable
  \texttt{processors}. Note that depending on your MPI installation,
  MPI jobs can only be run in the queueing system, so that \es{} will
  not run from the command line. In that case, you may not be able to
  run the testsuite, or you have to directly submit the testsuite script
  \verb!testsuite/test.sh! to the queueing system.\\
  \textbf{Example:} \verb!make check tests="madelung.tcl" processors="1 2"!\\
  will run the test \texttt{madlung.tcl} on one and two processors.
\item[\texttt{clean}] Deletes all files that were created furing the
  compilation.
\item[\texttt{mostlyclean}] Deletes most files that were created
  during the compilation. Will keep for example the built doxygen
  documentation and the \es{} binary.
\item[\texttt{dist}] Creates a \texttt{.tar.gz}-file of the \es{}
  sources.  This will include all source files as they currently are
  in the source directory, \ie{} it will include local changes.  This
  is useful to give your version of \es{} to other people.
  The variable \texttt{extra} can be used to specify additional
  files and directories that are to be included in the archive
  file. \\
  \textbf{Example:} \verb!make dist extra="myconfig.h internal"!\\
  will create the archive file and include the file
  \texttt{myconfig.h} and the directory \texttt{internal} with all
  files and subdirectories.
\item[\texttt{install}] Install \es{}. The variables \texttt{prefix}
  and \texttt{exec-prefix} can be used to specify the installation
  directories, otherwise the defaults defined by the
  \texttt{configure} script are used. \texttt{prefix} sets the basic
  prefix where all \es{} files are to be installed,
  \texttt{exec-prefix} sets the prefix where the executable files are
  to be installed and is required only when there is an
  architecture-specific directory.\\
  \textbf{Example:} \verb!make install prefix=/usr/local!\\
  will install all files below \texttt{/usr/local}.
\item[\texttt{uninstall}] Uninstalls \es{}, \ie{} removes all files
  that were installed during \texttt{make install}. The variables are
  identical to the variables of the \texttt{install}-target.
\end{description}

\section{Running \es}
\label{sec:run}

\es{} can be run via
\begin{syntax}
$>$ Espresso [\var{tcl\_script} [\var{N\_processors} [\var{args}]]]
\end{syntax}

\index{interactive mode} When \es{} is called without any arguments,
it is started in the interactive mode, where new commands can be
entered on the command line. When the name of a \textit{tcl\_script}
is given, the script is executed. \textit{N\_processors} is the number
of processors that are to be used. Any further arguments are passed to
the script. Note that depending on your MPI installation, MPI jobs can
only be run in the queueing system, so that \es{} will not run from
the command line.

% A number of wrapper scripts are used in running \es{}:
% \begin{itemize}
% \item The script \texttt{Espresso} in the source and build directory
%   will try to run the compiled version of \es. If it is called from
%   the source directory, it assumes that \es{} was also configured in
%   the source directory and will try to recursively start the script in
%   the corresponding \texttt{obj-PLATFORM} build directory. If it is
%   called in the build directory, it will start the \es-binary with the
%   right MPI implementation.
% \item The chooser script \texttt{Espresso} 
%   \begin{itemize}
%   \item installed when \verb!--enable-chooser! was given
%   \item installed to bindir
%   \item tries to run the correct version of the MPI-wrapper
%     \texttt{Espresso}
%   \end{itemize}
% \item The MPI-wrapper \texttt{Espresso}
%   \begin{itemize}
%   \item installed next to \es{} binary
%   \item starts the binary with the right MPI implementation
%   \end{itemize}
% \item The \es{} binary \texttt{Espresso-bin} can also be started
%   directly, however, it requires that the environment variable
%   \verb!ESPRESSO_SCRIPTS! is set to the directory where the scripts
%   are installed (usually \verb!$(prefix)/lib/espresso/scripts! or
%   \verb!$(prefix)/share/espresso/scripts!).
% \end{itemize}



%%% Local Variables: 
%%% mode: latex
%%% TeX-master: "ug"
%%% End: 
